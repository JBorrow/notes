\documentclass[11pt, a4paper]{report}
\usepackage{../../general_style}
\begin{document}
\title{Mathematics Workshop}
\author{Josh Borrow \\ \emph{Durham University}}
\date{\today}
\maketitle

\chapter{Complex Analysis}

\section{Basics}

$\mathbb{R}$, the set of real numbers, is a field:
\begin{itemize}
	\item We can add
	\item Also a group (if we remove $0$) under multiplication
	\item Distributive
\end{itemize}
However, not all simple equations have a solution, for example $x^2 + 1 = 0$.
We define $i^2 = -1$ to extend the real numbers,
$$
	z = x+iy; \;\;\;x,y\in\mathbb{R}~.
$$
We call this space/field $\mathbb{C}$.
We can already see that $\mathbb{C}\approx\mathbb{R}^2$ which gives rise to the argand diagram.

$x-iy$ is called the complex conjugate of $z$, $z^*$. We write
$$
	z z^* = |z|^2~.
$$

\subsection{Polars}
We can write $z$ as
$$
	z = re^{i\theta}~,
$$
and the complex conjugate simply replaces $\theta \rightarrow - \theta$.

\subsection{Complex Functions}
How do we deal with multi-rooted functions?
Example:
$$
	\sqrt{z} = \pm r e^\frac{i\theta}{2}~.
$$
We need to specify a choice of root: \emph{branch choice} or \emph{branch cut} to keep the functions single valued.
This comes into play when doing complex integration.

\subsection{Logarithms and Exponentials}
We define the exponential function using its taylor series.
Hence the logarithm:
$$
	\log(e^z) = z~,
$$
$$
	\log(re^{i\theta}) = \log r + i\theta~.
$$
Be careful: $\log$ is multi-valued!
We ened to do a branch cut to find the principal values:
$$
	\log(z) = \log(|z|) + i\arg(z) \; \; \; \arg(z) \in (-\pi, \pi]~.
$$
To write a general exponent
$$
	a^z = \exp(z\ln(a))~.
$$

\subsection{Trigonometric and Hyperbolic Functions}
$$
	\cos\theta = \frac{e^{i\theta} + e^{-i\theta}}{2}~,
$$
can be extended easily for complex numbers.
However, for complex numbers, $\cos\theta$ is not bounded between $|\cos\theta| \leq 1$.

\section{Complex Differentiation and Cauchy-Riemann}

\subsection{Continutiy}
Continuity meanst aht the closer we get to some limiting $z_0$ the closer we get to the value of the function at that point (i.e. the function doesn't jump around).
$$
	\lim_{z\rightarrow z_0} f(z) = \omega~~,
$$
$$
	\iff ~ \forall ~ \epsilon > 0 ~ \exists ~ \delta > 0 ~ s.t. ~ | z - z_0 | < \delta ~ \rightarrow ~ |f(z) - \omega|<\epsilon~.
$$
$f$ is continuous at $z_0$ if
$$
	\lim_{z\rightarrow z_0} f(z) = f(z_0)~.
$$

\subsection{Differentiation}
$$
	\deriv{f}{x} = \lim_{h\rightarrow 0} \frac{f(x+h) - f(x)}{h}~.
$$
Can we apply this to complex numbers?
It turns out, yes, but that the function is only differentiable if the above limit is independent of $\arg(h)$.

For example, we find
$$
	\deriv{z^*}{z} = \lim_{\delta \rightarrow 0} \frac{\z^* + \delta^* - \z^*}{\delta} = \lim_{\delta \rightarrow 0} \frac{\delta^*}{\delta}~.
$$
As this result depends on $\arg\delta$, we know that $z^*$ is not differentiable over the whole space.

\subsection{Analytic Functions}
Analytic functions are differentiable in at least one region.

If a function is differentiable everywhere, then it is \emph{entirely} differentiable.

\subsection{Cauchy-Riemann}
We can write
$$
	f(z) = u(x,y) + iv(x,y)~.
$$
If we would like to find the differfential 

\end{document}
