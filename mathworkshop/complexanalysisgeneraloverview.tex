\documentclass[11pt, a4paper]{Article}
\usepackage{../../general_style}
\begin{document}

\title{Complex Analysis}
\author{Josh Borrow}
\date{\today}

\maketitle

\section{Lecture 1}

The set of real numbers is a field.

Introduce $i = \sqrt{-1}$, the imaginary unit.

Write complex numbers
$$
	x+iy = re^{i\theta}~,
$$

Write complex functions
$$
	f(z) = u(x,y) + iv(x,y)~.
$$

Branch cuts are important to prevent integration over multi-valued regions.

Define $e^z$ through its Taylor expansion
$$
	e^z = \sum_{n=0}^\infty \frac{z^n}{n!}~.
$$

Logarithm is then
$$
	\ln(z) = \ln(r) + i\theta~.
$$
which is multi-valued so we branch cut
$$
	\ln(z) = \ln(|z|) + i \arg(z)~.
$$

Trigonometric functions are defined as before, but note that they are no longer bounded by unity.

\section{Lecture 2}

A function is continuous if as we get closer to $z$ in $f(z)=w$, we get closer to $w$ not further away.

Complex differentiation is defined in the same way as real differentiation,
$$
	\deriv{f}{z} = \lim_{\delta \rightarrow 0}\frac{f(z + \delta) - f(z)}{\delta}~.
$$
However, we must have that this is independent of $\arg{\delta}$.

An analytic function is one that is differentiable in at least some region.
If the function is differentiable everywhere then it is called entire.

The Cauchy-Riemann equations:
$$
	\pderiv{u}{x} = \pderiv{v}{y}~, ~~~\pderiv{v}{x} = -\pderiv{u}{y}~,
$$
for $f(z) = u(x,y) + iv(x,y)$. These equations imply that $\nabla^2 u = \nabla^2 v = 0$.

If $f$ is analytic on $\mathbb{C}$ then $u$ and $v$ are harmonic functions. 

All of this implies that a function is analytic so long that
$$
	\pderiv{f}{\bar{z}} = 0~,
$$
where $\bar{z}$ is treated as being independent of $z$.

\section{Lecture 3}

An open set $G\subset\mathbb{C}$ is one where you can always draw a circle around a point and have this circle be inside $G$.

$F$ is a closed set if $\mathbb{C}\backslash F$ is open.

A continuous curve $\gamma$ in $\mathbb{C}$ is a map from the real line to the complex plane.

An open set is split if we can write $G = G_1 \cup G_2$ such that $G_1 \cap G_2 = \phi$.

$G$ is connected if $G$ does not split.
$G$ is simply connected if any pair o curves between two points in $G$ can be deformed into each other without leaving $G$, i.e. $G$ has no holes.

Complex integration is path dependent. 
Generally, we need to paramaterize the curve.
$$
	\int_\gamma f(z)\mathrm{d}z = \int_\gamma u+iv(\mathrm{d}x + i\mathrm{dy})~.
$$

Cauchy's theorem states that the integral of an analytic function is ptah independent.

This implies that, for $f$ analytic over the reigon enclosed by $\gamma$,
$$
	\oint_\gamma f(z)\mathrm{d}z = 0~.
$$
We can prove this using Cauchy-Riemann inequalities.

\section{Lecture 4}

Cauchy's Integral Formula:
$$
	f(z_0) = \frac{1}{2\pi i} \oint_C \frac{f(z)\mathrm{d}z}{z-z_0}~,
$$
where $f$ is analytic on $D$, the reigion enclosed by $C$. Then
$$
f^n(z_0) = \frac{n!}{2\pi i} \oint \frac{f(z)}{(z-z_0)^(n+1)} \mathrm{d}z~.
$$

We can use this integral formula to find Taylor series,
$$
	f(z) = \sum_{n=0}^\infty \frac{f^n(z_0)}{n!}(z-z_0)^n~.
$$


\section{Lecture 5}

Holomorphic functions can be differentiated and integrated infinitely many times.

Louville Theorem: every bounded entire analytic function is a constant.

A holomorphic function $f$ has a zero of order $n$ at $z_0$ if we can write
$$
	f(z) = (z-z0)^ng(z)~,
$$
where $g$ is holomorphic in $D$, and $g(z_0) \neq 0$.

Let $f(z)$ be analytic in a domain $D \backslash \{z_0\}$, then $f$ has a pole at $z_0$ of order $n$ if we can write
$$
	f(z) = \frac{g(z)}{(z-z_0)^n}~,
$$
where $g$ is holomorphic on $D$ and nonzero at $z_0$.

This leads to a general theorem about contour integration
$$
	\oint f(z) \mathrm{d}z = 2\pi i \lim_{z\rightarrow z_0} (z-z_0)f(z)~.
$$

\section{Lecture 6}

A function is memomorphic if it is holomorphic except it has poles, i.e. it can be written
$$
	f(z) = \frac{g(z)}{(z-z_0)^n}~,
$$
g(z) holomorphic. Because g(z) has a taylor series, we must be able to write f(z) as a taylor series with negative powers of $(z-z_0)$.

If $f$ has a pole, then the taylor series termiantes at some finite negative $n$.
If it does not, then then it has an essential signularity. This function takes all values except 1 in the neighborhood of the origin.

Why do we care? $a_{-1}$ is called the residue of $f(z)$ and so we can 
$$
	\oint_C f(z) \mathrm{d}z = a_{-1} 2 \pi i~.
$$

Finding residues
\begin{itemize}
	\item Simple poles: $Res(f, z_0) = \lim_{z\rightarrow z_0} (z-z_0) f(z)$
	\item Poles of order $n$, $\frac{1}{(n-1)!} g^{(n-1)}(z_0)$
	\item Taylor expansion.
\end{itemize}

The Residue theorem states
$$
	\oint_c f(z) \mathrm{d}z = 2\pi i \sum_{i\in \mathbb{I}} Res(f, z_i)~.
$$

\section{Lecture 7&8}

Theorem: If $f$ is meromorphic on $\mathbb{C}$, then
$$
	\frac{1}{2\pi i} \oint \frac{f'(z)}{f(z)}\mathrm{d}z = n_{zeros} - n_{poles}~.
$$

We can rewrite real, trigonometric integrals as complex ones over the unit circle
$$
	z = e^{i\theta}; ~ ~ ~ \mathrm{d}z = iz\mathrm{d}\theta 
$$
This sometimes makes them easier.

Complex contour integration can also make real integrals easier, by integrating over a closed path.
$$
	\int^\infty_{-\infty} f(x)\mathrm{d}x = 2\pi i (Res_C) + \pi i (Res_\mathbb{R})~.
$$

\section{Lecture 9}

We define the Riemann-Zeta function
$$
	\eta(s) = \sum^\infty_{n=1} n^{-s}~.
$$
We can find this by using trigonometric functions and poles.
\end{document}
